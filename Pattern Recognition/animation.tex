\documentclass{beamer}
\usepackage{animate}
\usepackage{color}
\usetheme{warsaw}

\usepackage{graphicx}
\usepackage{fancyvrb}
\usepackage[3D]{movie15}
\usepackage{multimedia}

\begin{document}


\begin{frame}
%\transpush
\frametitle{\onslide<1>{PATTERN RECOGNITION}}
\transdissolve<1>

\begin{itemize}
\item \onslide<2>{Pattern recognition is the procedure of processing and analizing diverse information(numerical,literal,logical) characterizing the objects or phenomenon, so as to provide descriptions, identification, classification and interpreations for them.}\transdissolve<2>
 \item  \onslide<3>{{\bf``\underline {Perceive + Process +Prediction}" :-} It is the study of how machine can}\transdissolve<3>
 \begin{itemize}
\item  \onslide<4>{[$\diamond$] Perceive: Observe the environment(i.e. Interact with the real-world).}\transdissolve<4>
 \item \onslide<5>{[$\diamond$]Process: Learn to distinguish pattern of interest from their background.} \transdissolve<5>
  \item \onslide<6>{[$\diamond$] Prediction: Make sound and reasonable decisions about the categories of the pattern.}\transdissolve<6>
 \end{itemize}
\end{itemize}
\end{frame}


\begin{frame}
%\transdissolve

\begin{columns}


\begin{column}{.35\textwidth}
\includegraphics[width=.35\textwidth]{mlnc.png}%{\hspace{1.5cm}} \pause
\end{column}


\begin{column}{.6\textwidth}
{\bf \underline{MOTILAL NEHRU COLLEGE }}\\ \hspace{.4cm}{\bf \underline{ UNIVERSITY OF DELHI }}
\end{column}


\begin{column}{.4\textwidth}
\includegraphics[width=.4\textwidth]{du.jpg}%{\hspace{1cm}} %\pause
\end{column}


\end{columns}
\hspace{2.2cm} {\bf \underline{DEPARTMENT OF MATHEMATICS }}\\









 \end{frame}




\begin{frame}

%
\href{run:vid.mp4}{\includegraphics[scale=.4]{mlnc.png}}



\end{frame}
\begin{frame}
\transwipe
    \only<1>{Test 1}
    \only<2>{Test 1\alert{\ldots}}
    \only<3>{Test 1\ldots 2}
    \only<4>{Test 1\ldots 2\ldots}
    \only<5>{Test 1\ldots 2\ldots 3}
    \only<6>{Test 1\ldots 2\ldots 3\ that's all folks}

    \transdissolve<2>
    \transblindshorizontal<3>
    \transboxout<4>
    \transglitter<5>
    \transwipe[direction=90]<6>

    %
\href{run:na.mp3}{\includegraphics[scale=.4]{du.jpg}}

\end{frame}

\begin{frame}
\transdissolve
\begin{columns}
\begin{column}{.5\textwidth}
This is college logo.
\end{column}
\begin{column}{.5\textwidth}
    \only<1>{
    \begin{figure}
    \includegraphics[width=.5\textwidth]{mlnc.png}
    \end{figure}}
    \only<2>{
    \begin{figure}
    \includegraphics[width=.5\textwidth]{du.jpg}
    \end{figure}}
    \only<3>{
    \begin{figure}
    \includegraphics[width=.5\textwidth]{fface.jpg}
    \end{figure}}
    \only<4>{Nature Recognition\ldots 2\ldots}
    \only<5>{PR\ldots 2\ldots 3}
    \only<6>{PR 1\ldots 2\ldots 3\ that's all folks}

    \transdissolve<2>
    \transblindshorizontal<3>
    \transboxout<4>
    \transglitter<5>
    \transwipe[direction=90]<6>
\end{column}
\end{columns}

\end{frame}

\begin{frame}
\transpush
\frametitle{PATTERN}
\begin{itemize}
\item A Pattern is a set of objects or phenomena or concepts where the element of the set are similar to one another in certain ways or aspects. \pause
\item A Pattern is an entity, that could be given a name. \pause
\begin{itemize}
\vspace{1cm}\item[Example:-]
\end{itemize}
\end{itemize}


\begin{columns}

\begin{column}{.3\textwidth}
                \only<1>{
                 \begin{figure}
                  \includegraphics[width=.3\textwidth]{finger.jpg}
                   \caption {Fingerprint}
                    \end{figure}}

                      \only<2>{
                       \begin{figure}
                        \includegraphics[width=.3\textwidth]{finger1.jpg}
                         \caption {Fingerprint}
                          \end{figure}}

                            \only<3>{
                             \begin{figure}
                              \includegraphics[width=.3\textwidth]{finger2.jpg}
                               \caption {Fingerprint}
                                \end{figure}}



                 \transglitter<2>
                  \transglitter<3>

\end{column}


\begin{column}{.3\textwidth}
\begin{figure}
\includegraphics[width=\textwidth]{hand.jpg}
\caption {Handwitten}
\end{figure}
\end{column}
\begin{column}{.3\textwidth}
\begin{figure}
\includegraphics[width=\textwidth]{dna.jpg}
\caption {DNAsequence}
\end{figure}
\end{column}
\end{columns}


\end{frame}

\begin{frame}
\begin{columns}
\begin{column}{.3\textwidth}
\begin{figure}
\includegraphics[width=\textwidth]{hand.jpg}
\caption {Handwitten}
\end{figure}
\end{column}
\begin{column}{.3\textwidth}
\begin{figure}
\includegraphics[width=\textwidth]{dna.jpg}
\caption {DNAsequence}
\end{figure}
\end{column}
\begin{column}{.3\textwidth}
%\includemovie[autoplay,repeat]
%{4cm}{3cm}{jstar.mp4}
\end{column}
\end{columns}


\end{frame}

\end{document}